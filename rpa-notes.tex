\documentclass[12pt,a4paper,english]{article}
\usepackage{ucs}
\usepackage[latin1]{inputenc}%utf-8 utf8x
%\usepackage{lmodern}
\usepackage{fontenc}%[T1]
\usepackage{babel}%[T1]
\usepackage{amssymb,amsmath,wick}
\usepackage{epsfig}
\usepackage{wrapfig}
\usepackage{color,subfigure}  
%\usepackage{beamerthemesplit}
\usepackage{graphicx}
\usepackage{listings}
%\usepackage[pdftex,colorlinks=true,bookmarks=true,linkcolor=blue]{hyperref}
\let\Tiny=\tiny

\newcommand{\be}{ \begin{equation}}
\newcommand{\ee}{ \end{equation}}
\newcommand{\mc}{ \mathcal }
\newcommand{\mbf}{ \mathbf }
\newcommand{\bra}[1]{\langle #1|}
\newcommand{\ket}[1]{|#1\rangle}
\newcommand{\braket}[2]{\langle #1|#2\rangle}
\newcommand{\braopket}[3]{\langle #1|#2|#3\rangle}
\newcommand{\beq}{\begin{equation*}}
\newcommand{\eeq}{\end{equation*}}
\newcommand{\ds}{\displaystyle{\not}}
\newcommand{\matr}[1]{{\bf \cal{#1}}}
\newcommand{\OP}[1]{{\bf\widehat{#1}}}

\newcommand{\clearemptydoublepage}{\newpage{\pagestyle{empty}\cleardoublepage}}


\title{My Notes for the random phase approximation}
\begin{document}
\section{Introuduction}
The random phase approximation (rpa) was introduced by Bohm and Pines in \cite{Bohm1951,Pines1952,Bohm1953}. It was introduced in relation to describe 
the collective behaviour of electrons in electron interactions.
\emph{The random phase comes from the two kinds of response of 
the electrons to a
wave. ``One of these is in phase with the wave, so that the phase difference
between the particle responce and the wave producing it is independent of
the position of the particle. This is the response which contributes to the 
organized behavior of the system. The other response has a phase difference
with the wave producing it which dpends on the position of the particle.
Because of the general random location of the particles, this second response 
tends to average out to zero when we consider a large number of electrons, and
we shall neglect the contribution arising from this. This procedure we call the
random phase approximation.}''\cite{Bohm1951}





\bibliographystyle{h-physrev3}
%\bibliographystyle{unsrt}
\bibliography{rpa.bib}

\end{document}
